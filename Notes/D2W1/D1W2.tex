% File:      NotesTemplate.tex
% Author:    Gage Sorrell <gsorrell@purdue.edu>
% Copyright: (c) 2024 Gage Sorrell
% License:   MIT

\documentclass[letterpaper,twoside]{article}

% Packages

\usepackage[english]{babel}          % Common package
\usepackage[utf8]{inputenc}          % Common package
\usepackage[margin=0.5in,bottom=0.75in]{geometry} % Page margins
\usepackage{amsmath}                 % General math
\usepackage{amssymb}                 % For defining \RealExpressions
\usepackage{graphicx}                % For `\includegraphics{}`
\usepackage{setspace}                % For `\onehalfspacing`
\usepackage{float}                   % Position options for `\includegraphics{}`
\usepackage{amsfonts}                % `\mathbb{}`, etc.
\usepackage{ragged2e}                % For justified text
\usepackage{tocloft}                 % For customizing the Table of Contents
\usepackage{xcolor}                  % To use colors
% \usepackage{tcolorbox}               % For boxed text
\usepackage{emoji}                   % For emoji
\usepackage{makecell}                % For linebreaks in table cells
\usepackage{fancyhdr}                % For the header and footer
\usepackage{multicol}                % Two-column layout
\usepackage{enumitem}                % Set margins within itemize environments
\usepackage{titlesec}                % For adjusting the space above and below section titles
\usepackage{tikz}                    % For unit vectors
\usepackage{mathtools}               % For :=

% Variables

\def\Author{Gage Sorrell}
\def\BulletPointSeparator{\SmallHSpace$\cdot$\SmallHSpace}
\def\CourseName{MA 15300-02}
\def\CourseNameFriendly{College Algebra}
\def\DocumentTitle{MA153 Notes Day \LectureDay Week \LectureWeek}
\def\LectureDay{2}
\def\LectureWeek{1}
\def\SmallHSpace{\hspace*{1mm}}

% Commands

\newcommand{\Eg}[1]{\textit{e.g.}, #1}
\newcommand\Ie[1]{\textit{i.e.}, #1}

% Define the command for a stylized definition
% #1 - Term to be defined
% #2 - Definition of the term
\newcommand{\DefinedTerm}[1]{\textbf{#1}}
\newcommand{\Definition}[1]{%
    % \begin{tcolorbox}[enhanced jigsaw, % Allows box to break across pages
    %                 %   colback=blue!5, % Background color
    %                   colback=white, % Background color
    %                   colframe=blue!75!black, % Frame color
    %                   arc=0mm, % Removes rounded corners
    %                   boxrule=0.5pt, % Frame thickness
    %                   top=2pt, bottom=2pt, left=5pt, right=5pt] % Spacing around the text
    % \emoji{book} \textbf{Definition.}\SmallHSpace #1
    % \end{tcolorbox}
    \emoji{book} \textbf{Definition.}\SmallHSpace #1 \hfill $\square$
}

\newcommand{\Note}[1]{%
    \emoji{warning} \textbf{Note.}\SmallHSpace #1 \hfill $\square$
}

\newcommand{\Paradox}[1]{%
    \emoji{police-car-light} \textbf{Paradox.}\SmallHSpace #1 \hfill $\square$
}

\newcommand{\Example}[1]{%
    \emoji{magnifying-glass-tilted-right} \textbf{Example.}\SmallHSpace #1 \hfill $\square$
}

\newcommand{\Theorem}[1]{%
    \emoji{thinking-face} \textbf{Theorem.}\SmallHSpace #1 \hfill $\square$
}

\newcommand{\Proof}[1]{%
    % \begin{tcolorbox}[enhanced jigsaw, % Allows box to break across pages
    %                 %   colback=blue!5, % Background color
    %                   colback=white, % Background color
    %                   colframe=blue!75!black, % Frame color
    %                   arc=0mm, % Removes rounded corners
    %                   boxrule=0.5pt, % Frame thickness
    %                   top=2pt, bottom=2pt, left=5pt, right=5pt] % Spacing around the text
    % \emoji{book} \textbf{Definition.}\SmallHSpace #1
    % \end{tcolorbox}
    \emoji{brain} \textbf{Proof.}\SmallHSpace #1 \hfill $\square$
}

\newcommand{\Lemma}[1]{%
    % \begin{tcolorbox}[enhanced jigsaw, % Allows box to break across pages
    %                 %   colback=blue!5, % Background color
    %                   colback=white, % Background color
    %                   colframe=blue!75!black, % Frame color
    %                   arc=0mm, % Removes rounded corners
    %                   boxrule=0.5pt, % Frame thickness
    %                   top=2pt, bottom=2pt, left=5pt, right=5pt] % Spacing around the text
    % \emoji{book} \textbf{Definition.}\SmallHSpace #1
    % \end{tcolorbox}
    \emoji{thinking-face} \textbf{Lemma.}\SmallHSpace #1 \hfill $\square$
}

\newcommand{\Tldr}[1]{%
    % \begin{tcolorbox}[enhanced jigsaw, % Allows box to break across pages
    %                 %   colback=blue!5, % Background color
    %                   colback=white, % Background color
    %                   colframe=blue!75!black, % Frame color
    %                   arc=0mm, % Removes rounded corners
    %                   boxrule=0.5pt, % Frame thickness
    %                   top=2pt, bottom=2pt, left=5pt, right=5pt] % Spacing around the text
    % \emoji{book} \textbf{Definition.}\SmallHSpace #1
    % \end{tcolorbox}
    \emoji{stopwatch} \textbf{Summary.}\SmallHSpace #1 \hfill $\square$
}

% Configuration

\author{\Author}
\onehalfspacing
\setlength{\parindent}{0pt}
\title{\DocumentTitle}

% TOC
\renewcommand{\cftsecleader}{\cftdotfill{\cftdotsep}}
\tocloftpagestyle{empty}

% \tcbuselibrary{skins,breakable} % Extra features for tcolorbox
\setlength{\parskip}{3mm}
\setlength{\columnsep}{0.5in}
\titlespacing*{\section}{0pt}{4pt}{0pt}
\titlespacing*{\subsection}{0pt}{4pt}{0pt}
\titlespacing*{\subsubsection}{0pt}{4pt}{0pt}

\pagestyle{fancy}
\fancyhf{}
\setlength{\headwidth}{7.25in}
\fancyhead[L]{\textsc{\CourseName}}
\fancyhead[R]{\textsc{Day} \LectureDay \ \textsc{Week} \LectureWeek}
\renewcommand{\footrulewidth}{0.4pt}
\renewcommand{\headrulewidth}{0.4pt}
\fancyfoot[L]{\textsc{Purdue University Fort Wayne}}
\fancyfoot[R]{Page \thepage}

% Environments

\newenvironment{SorrellEnumerate}
{
    \setlength\parskip{-5pt}
    \begin{enumerate}
        \setlength\itemsep{-4pt}
}{
    \end{enumerate}
}

\newenvironment{SorrellItemize}
{
    \setlength\parskip{-5pt}
    \begin{itemize}[leftmargin=11pt]
        \setlength\itemsep{-4pt}
}{
    \end{itemize}
}

% END PREAMBLE %

\begin{document}

\begin{titlepage}
    \newgeometry{margin=1.25in}
    \begin{figure}[t]
        \centering
        \includegraphics[width=1.5in]{../../Resources/Letterhead.png}
    \end{figure}
    \vspace*{0.5in}
    \begin{center}
        \huge
        \textsc{Lecture Notes}
        
        \normalsize
        \CourseName \BulletPointSeparator \Author

        \vspace*{0.25in}
        \large
        \begin{table}[H]
            \centering
            \begin{tabular}{ll}
                \textsc{Lecture} \textnumero & \textsc{day 1 week 1}              \\
                \textsc{Date}                & \textsc{january 9, 2024 (tuesday)}
            \end{tabular}
        \end{table}
        \begin{table}[H]
            \centering
            \begin{tabular}{ll}
                \textsc{Topics} & \textsc{Textbook Section(s)}\\
                \hline
                Relations      & \textbf{1.1} \\
                Functions      & \textbf{1.1} \\
                Curves         & \text{1.1}\\
                Rate of Change & \text{1.2}\\
            \end{tabular}
        \end{table}
        \vspace*{\fill}
        \footnotesize
        \justifying
        \leftskip=0.75in
        \rightskip=0.75in
        \paragraph*{Objective}
        The concept of the relation is expanded upon, and the function is formalized.
        Our focus shifts toward computation, and geometric intuition by constructing curves from functions.
        The notion of \textit{rate} is introduced, and it is shown how linear functions use rates to model linear phenomena in the natural world.
    \end{center}
\end{titlepage}

\newpage
\newgeometry{margin=1.5in}
\tableofcontents
\newpage
\restoregeometry
\newgeometry{inner=0.75in,outer=0.5in,top=0.75in,bottom=0.75in}
\setcounter{page}{1}

\begin{multicols*}{2}

% \section{Syllabus Q \& A}

% Before beginning the lecture, we will do the following:

% \begin{SorrellEnumerate}
%     \item Instructor introduction
%     \item Take any questions regarding the syllabus
%     \item Highlight key parts of the syllabus
%     \item Perform a demonstration of generative AI for learning course content
% \end{SorrellEnumerate}

\section{The Relation}

Recall that the relation was defined in the last lecture.

\subsection{The Graph}

As briefly stated in the last lecture, \textit{graphs} are pictorial representations of relations.
Graphs are useful, particularly for relations defined on countable sets (finite or infinite).

For a picture to be a graph, it just needs to demonstrate a relation in some well-defined way.
Often, graphs are drawn for a subset of the set that its relation is defined on, with just enough drawn such that the pattern (the ``gist'') of the relation is demonstrated.
This is convenient for large countable sets, and is necessary for infinite sets.

Graphs are typically drawn by drawing each element as a dot, and each assignment as an arrow pointing from one element to the other.
The dot is called the \textit{node} or \textit{vertex}, and the arrow is called the \textit{edge}.

Sometimes, if the set that the relation is defined on has subsets that are meaningful in the context of the relation, we choose to draw each subset as a loop that contains the nodes of that subset.

@TODO Two examples

\subsection{An Important Consequence from ZFC}

We are going to define the function as a relation, but before we do, we must observe an important result from ZFC.
This result will allow us to form a definition of a specific type of relation, to which the function belongs.

Recall that by ZFC, for every set S, we can define subsets $S_0$ and $S_1$ that contain any elements of $S$.
Additionally, regardless of what is in $S_0$ and $S_1$, the union of $S_0$ and $S_1$ is guaranteed to exist.

Therefore, for any set $S$, we can define $S_0$ and $S_1$ such that they are \textit{disjoint} (they do not share any elements) and their union is $S$.

Summarized,
\begin{align}
    \forall S : S \text{ is a set}, \exists S_0, S_1 : S_0 \cup S_1 = S, S_0 \cap S_1 = \emptyset
\end{align}

@TODO Example

\subsection{The Functional Property}

Additionally, we looked at the equivalence relation, which is a relation that has three specific properties.
We also looked at \textit{order}, which is another relation.

There are many other properties that relations can have.
One more important property that a relation can have is the property of being \textit{functional}

\Definition
{
    A relation is \DefinedTerm{functional} if, for disjoint subsets $S_0$ and $S_1$, every assignment is \textit{from} $S_0$ \textit{to} $S_1$, or (equivalently), from $S_1$ to $S_0$.
}

Functional relations rely on the \textit{partition} (the disjoint subsets) to define a direction to the relation as a whole.

@TODO Example

\subsection{The Function}

\Definition
{
    The \DefinedTerm{function} is a relation that is functional and total.
}

Functions are the basic object that allow us to model \textit{transformation}, by \textit{accepting} an object, and \textit{returning} an output, whose value depends upon the input.

@TODO Example

@TODO Example of many-to-many, one-to-many, many-to-one

Functions are defined for an \textit{input} variable, which is also called the \textit{argument}.

The set of elements that the assignments are \textit{from} is the \textbf{domain}.

The set of elements that the assignments are \textit{to} is the \textbf{codomain}.
Because elements can be assigned to the same element, the set of elements that are assigned \textit{to} is not necessarily the codomain, but a subset of the codomain.
This set of elements that are assigned \textit{to} is called the \textbf{range}.

\subsubsection{Notation}

To represent a function, we often state its name, followed by a comma-delimited list which is wrapped in parentheses.

\Note
{
    Because mathematicians like to be concise, they often use one-letter names to represent a function, especially when making general statements about functions.
    When you are solving problems modeled after real-world phenomena, you are encouraged to come up with descriptive names for your functions.
    You might find that this makes your work more organized.
}

When defining a function in formal notation, we often use the equality symbol between the name of the function, and an expression that defines the function.
The argument list is needed so that the arguments can be identified when reading/writing the expression that defines the function.

@TODO Examples

@TODO For this class, we'll stick to some common names for domain, elements in the domain, etc.

@TODO Define image, pre-image

\subsubsection{The Rule of Four}

The ``rule of four'' is not a rule in a strict sense, but is the observation that any well-defined function can be stated in four different ways.

If you have defined a function, but struggle to formulate an equivalent definition in one of the other forms, that is in indicator that your definition might not be well-formulated and consistent.

@TODO Definition and example

\subsubsection{Injection}

In addition to being functional and total, some functions assign \textit{to} elements once, or never.
When a function never assigns multiple elements to the same element, it is \textit{injective} (equivalently, the function is an \textit{injection}).

\Definition
{
    Formally, this is stated as,
    \begin{align}
        \forall x_i, x_j \in X, f(x_i) = f(x_j) \Longrightarrow x_i = x_j
    \end{align}
}

Injections guarantee \textit{uniqueness}.

@TODO Example

\subsubsection{Surjection}

Some functions assign \textit{to} every element in the codomain.
As a result @TODO

\Note
{
    Regardless of whether the codomain is the range for a given function, the codomain is still a proper subset of the range.
    This is how we define subsets: a set is the subset of another set if the intersection of the two sets is non-empty.
}

Surjections guarantee \textit{existence} of a pre-image for every element in the codomain.

@TODO Example

\subsubsection{Bijection \& Invertability}

\Definition
{
    When a function is injective \textit{and} surjective, it is \DefinedTerm{bijective}.
}

Bijections give some form of ``same-ness'' (equality) to the elements by the assignments that are made among them.

Because surjection guarantees that we can go ``the other way,'' and injection guarantees that going ``the other way'' does not give a one-to-many relation, the function is a \textit{one-to-one} relation.

@TODO Two Examples, use programming experience to define an example

\section{Rate}

\Definition
{
    \DefinedTerm{Rate} models phenomena in which one quantity is manipulated as a function of another quantity.
}

Because this is a tool that specifically describes the natural world, it is best to simply look at many examples, 
\begin{SorrellItemize}
    \item a plane uses some gallons of fuel per mile of travel
    \item safe consumption of caffeine in one day (for most people) is 3 to 6 \textit{mg} per \textit{kg} of bodyweight
    \item for every second of travel, an object falling to the surface of the Earth falls an additional 9.8 meters per second
\end{SorrellItemize}

Rate is typically expressed as a ratio of units, such as $\frac{miles}{hour}$.

\section{The Linear Function}

\section{Curves of Functions}

% \Tldr
% {
%     This lecture covers material that you have seen before, \textit{and} foundational content that formally defines this material.
%     Some of this content will be material that you have seen before, but is stated in a way that you may not have seen before.
%     \textbf{Do not worry when you see material that is new to you.}
% }


\hfill\emoji{skull}
\end{multicols*}

\end{document}
